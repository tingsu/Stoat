%    Angulo.  Measure angles and slopes with Android!
%    Copyright (C) 2011  Daniel Kraft <d@domob.eu>
%
%    This program is free software: you can redistribute it and/or modify
%    it under the terms of the GNU General Public License as published by
%    the Free Software Foundation, either version 3 of the License, or
%    (at your option) any later version.
%
%    This program is distributed in the hope that it will be useful,
%    but WITHOUT ANY WARRANTY; without even the implied warranty of
%    MERCHANTABILITY or FITNESS FOR A PARTICULAR PURPOSE.  See the
%    GNU General Public License for more details.
%
%    You should have received a copy of the GNU General Public License
%    along with this program.  If not, see <http://www.gnu.org/licenses/>.

\documentclass{article}
\usepackage{amsmath}
\usepackage{amssymb}
\usepackage{url}
\usepackage{hyperref}

\newcommand{\abs}[1]{\ensuremath{\left| #1 \right|}}
\newcommand{\inner}[2]{\ensuremath{#1 \cdot #2}}
\newcommand{\cross}[2]{\ensuremath{#1 \times #2}}
\newcommand{\trace}[0]{\ensuremath{\text{tr} \;}}
\newcommand{\R}[0]{\ensuremath{\mathbb{R}}}
\newcommand{\Rthree}[0]{\ensuremath{\R^3}}
\newcommand{\SO}[1]{\ensuremath{SO\left(#1\right)}}

\renewcommand{\eqref}[1]{\hyperref[eq:#1]{Equation~\ref*{eq:#1}}}
\newcommand{\secref}[1]{\hyperref[sec:#1]{Section~\ref*{sec:#1}}}

\newcommand{\Angulo}[0]{\textbf{Angulo} }

\title{Angulo}
\author{Daniel Kraft}
\date{\today}

\begin{document}
\maketitle

%%%%%%%%%%%%%%%%%%%%%%%%%%%%%%%%%%%%%%%%%%%%%%%%%%%%%%%%%%%%%%%%%%%%%%%%%%%%%%%%
\section{Introduction}

\Angulo is a tool which utilizes the internal sensors (for acceleration and
magnetic field) of a mobile device to allow measuring angles with it.  We
assume that both acceleration (the Earth's gravitational field) and the
surrounding magnetic field are constant and fixed in space; thus if we
measure both at two times and compare the measured vectors (their directions
in particular), we can calculate how the device was rotated between the
measurements.

Thus you can set a ``reference point'' with the device pointing in one
direction, and rotate it into another one, and have the angle between both
directions calculated.  With this tool, one can for instance find out the
inclination of a slope or actually simply measure an angle between any two
given directions.

The code is free (released under the GPL), check it out at
\url{http://www.sourceforge.net/projects/angulo/}.  \Angulo runs on Google's
Android operating system (and obviously needs devices which actually
have the sensors used).  This document is not intended as an usage introduction,
for that see the built-in help text or the webpage at
\url{http://www.domob.eu/projects/angulo.php}.  Instead, I want to describe
the internal logic here (in particular, how the shown results are calculated).

\Angulo displays measured angles mainly in degrees; for small angles (and
in particular slopes) sometimes also percent are used --- where the percent
value corresponds to the tangens of the actual angle.  If the angle is small
(below 45 degrees), \Angulo as a utility feature displays the same angle
again in percent.

%%%%%%%%%%%%%%%%%%%%%%%%%%%%%%%%%%%%%%%%%%%%%%%%%%%%%%%%%%%%%%%%%%%%%%%%%%%%%%%%
\section{With a Single Sensor}
\label{sec:singleValue}

Assume for a moment that we only consider data from a single sensor, and let
it without loss of generality be acceleration.  If we
measure twice, we get two vectors $a, b \in \Rthree$.  $a$ corresponds to
the acceleration at the first measurement (which \Angulo calls the
``reference point''), $b$ to that at the second time (which is the ``current''
value updated continuously during runtime).

Now, if $\phi$ denotes the angle in-between the vectors $a$ and $b$,
elementary vector analysis implies that
\begin{equation}
\label{eq:abCosine}
\cos \phi = \frac{\inner{a}b}{\abs{a} \abs{b}}.
\end{equation}
Hence, using the arccosine we can easily calculate the angle between both
directions.  However, since the inner product is symmetric, the same angle
will be reported if we rotate in the opposite direction --- which is somewhat
counter-intuitive.  On the other hand, the cross-product $\cross{a}b$ is
\emph{anti}symmetric.  Thus we can set the sign of $\phi$ depending on the
direction of $n = \cross{a}b$; since there's no ``absolute reference'' however,
\Angulo simply uses the sign of the sum of all component of $n$ as the
assumed sign of $\sin \phi$ and thus $\phi$.
That way, if $a$ and $b$ are flipped (which corresponds to rotating in the
opposite direction), also $n$ is flipped and the sign will be opposite.  This
is of course an arbitrary choice (and a lot of other options would be possible),
but it makes \Angulo display a useful sign in most circumstances.  Just note
to be careful in particular with the sign; but usually, the interesting
quantity is anyways the magnitude of the angle and not its direction (since that
is most of the time apparent).

In addition to \eqref{abCosine}, another relation is
\begin{equation}
\label{eq:abSine}
\abs{\sin \phi} = \frac{\abs{\cross{a}b}}{\abs{a} \abs{b}}.
\end{equation}
If we combine \eqref{abCosine} and \eqref{abSine}, we get (except for
the sign of $\sin \phi$, but the same solution as above applies for the
angle's sign) an even more (numerically) accurate result for the angle.

The problem with this approach is that it only measures the angle between
the two actually measured vectors --- which may not always correspond to the
angle the device was actually rotated!  For instance, if you rotate it exactly
around the direction of acceleration (which usually means horizontally), the
measured value \emph{does not change at all} and consequently, \Angulo will
display $\phi = 0$ despite there obviously being a nonzero rotation.  Since
\Angulo actually displays the two angles calculated independently from
the direction of acceleration and magnetic field, one can sometimes mitigiate
this problem by choosing one of them over the other, depending on the axis
of rotation; but it may well be that \emph{none} is correct on its own.
See \secref{combined} for a way to combine both values into a single
measurement that does not have that problem.

%%%%%%%%%%%%%%%%%%%%%%%%%%%%%%%%%%%%%%%%%%%%%%%%%%%%%%%%%%%%%%%%%%%%%%%%%%%%%%%%
\section{Combined Angle}
\label{sec:combined}

As mentioned above, one gets a better estimate on the actual angle if the
measurements of both sensors are combined into a single value --- this is
what \Angulo displays in large, green writing on the bottom of the screen and
the value is supposed to be the most accurate estimate available in most
cases.  This works well as long as the directions of both quantities used
are not linearly dependent --- but since the acceleration is usually pointing
straight through the floor and the magnetic field corresponds to Earth's
magnetic field,
those vectors being linearly dependent means that the magnetic field is
also pointing straight up or down, which in turn only happens at the
magnetic poles.  Thus this situation is quite unlikely, and usually the
the assumption of linear independence is fulfilled.

To make it more precise, let $a$ and $b$ be the values of acceleration and
magnetic field at the ``reference measurement'', respectively.  Let further
$a$ and $b$ be linearly independent.  After the
device is rotated, we assume that the new directions are $R a$ and $R b$, where
$R$ is the three-dimensional rotation matrix corresponding to the movement of
the device.  We assume that the ``current'' vectors can be represented in this
way and clearly $R \in \SO3$.

\subsection{Rotation Angle}
%%%%%%%%%%%%%%%%%%%%%%%%%%%

Assume now that $R$ is known.  If we choose a suitable orthonormal
coordinate system (with the axis of rotation being the third basis vector),
the matrix representation of $R$ will be
\begin{equation*}
R' = \left(\begin{array}{ccc}
       \cos \phi & -\sin \phi & 0 \\
       \sin \phi & \cos \phi & 0 \\
       0 & 0 & 1
     \end{array}\right).
\end{equation*}
Since the trace of a matrix is invariant under basis changes, also
\begin{equation}
\label{eq:rCosine}
\trace R = 1 + 2 \cos \phi.
\end{equation}
It can further be shown (see for instance the thread at
\url{http://www.mathworks.com/matlabcentral/newsreader/view_thread/160945})
that additionally
\begin{equation}
\label{eq:rSine}
\abs{\left(\begin{array}{c}
  R_{32} - R_{23} \\
  R_{13} - R_{31} \\
  R_{21} - R_{12}
\end{array}\right)} = 2 \abs {\sin \phi}
\end{equation}
holds.
Taking \eqref{rCosine} and \eqref{rSine} together, we can again nicely find
$\phi$ itself (up to the sign, but again this is not clearly defined).  For
the sign, the same rule of thumb is used as in \secref{singleValue} --- namely
all (in total six) components of $\cross{a}{Ra}$ and $\cross{b}{Rb}$
are summed up, and the sign of this sum is used to adapt the sign of the
resulting angle.

\subsection{Finding the Matrix}
%%%%%%%%%%%%%%%%%%%%%%%%%%%%%%%

It remains now to actually \emph{find} the matrix $R$ from the four
vectors $a, b, Ra, Rb$ that are known.  Note that since $R$ is orthogonal
and in particular regular, $Ra$ and $Rb$ are also linearly independent.
As a first step, we can orthonormalize the vectors $a$ and $b$ and extend
them to a right-handed orthonormal basis $(a', b', c')$ of $\Rthree$:
\begin{equation}
\label{eq:abcBasis}
a' = \frac{a}{\abs{a}}, \;\;
\tilde{b} = b - (\inner{a'}b) \cdot a' \ne 0, \;\;
b' = \frac{\tilde{b}}{\abs{\tilde{b}}}, \;\;
c' = \cross{a'}{b'}
\end{equation}
Let $B$ be the matrix formed by those basis vectors as columns; then
$B \in \SO3$ and applying $B^{-1} = B^\top$ onto a vector represents it
in the basis $(a', b', c')$.

Now, the columns of the representation of $R$ in this basis are simply
the images under $R$ of the basis vectors.  But since $R$ is linear and
also orthogonal, \emph{it preserves norms and inner products}.  Thus,
we get $Ra'$, $Rb'$ and $Rc'$ by simply applying the recipe of
\eqref{abcBasis} onto $Ra$ and $Rb$ in the places of $a$ and $b$.
Let $B'$ be the matrix whose columns are the vectors $Ra'$, $Rb'$ and $Rc'$,
which are known this way.  Then the full matrix is simply:
\begin{equation*}
R = B' B^\top
\end{equation*}
Hence, we know all that is necessary for calculating the angle.

\end{document}
